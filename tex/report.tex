\documentclass{article}
% \usepackage{sbc-template}
\usepackage{graphicx}
\usepackage[utf8]{inputenc}
\usepackage[T1]{fontenc}
\usepackage{pgfplots}
\usepackage{pgfplotstable} 
\usepackage{titlesec}
\usepackage{lipsum}
\usepackage{authblk}
\usepackage{mathtools}

\begin{document}

\title{Recuperação de Informação - Máquinas de Busca na Web: Trabalho Prático 3}
\author{João Mateus de Freitas Veneroso}
\affil{Departamento de Ciência da Computação da Universidade Federal de Minas Gerais}

\maketitle

\section{Introdução}

Este trabalho descreve a implementação de um sistema de consultas com base no arquivo
invertido criado no trabalho prático 2. O sistema de consulta implementado faz 
consultas ordenadas por meio do modelo de espaço vetorial com a utilização do
anchor text e do page rank para melhoria dos resultados. A interface gráfica do sistema
de consultas foi implementada com o NodeJS e está disponível para consultas na
url visconde.latin.dcc.ufmg.br:8080. A versão mais recente do código pode ser obtida
em: https://github.com/jmfveneroso/inverted-index. Para instruções de compilação e uso, leia
o arquivo README.md. 

\section{Índice invertido}

Esta seção descreve brevemente os detalhes do índice invertido utilizado pelo sistema
de consultas. As páginas foram reindexadas com o mesmo sistema construído no trabalho
prático 2. A coleção completa tem 6.175.317 páginas da web no domínio ".br" ocupando
aproximadamente 477 GB. 

\begin{table}
\centering
\begin{tabular}{ |l|l| }
  \hline
  \multicolumn{2}{|c|}{Etapa de extração} \\
  \hline
  Tamando da coleção & ~477 GB \\
  Número de documentos & 6.175.317 \\
  Número de termos & 9.273.617 \\
  Tamanho do arquivo invertido & ~11.20 GB \\
  Tempo para carregar & 120 segundos \\
  \hline
\end{tabular}
\caption{Índice invertido}
\label{tab:inverted_index}
\end{table}

\section{Sistema de consultas}

As características do arquivo invertido estão descritas na 
tabela \ref{tab:inverted_index}. As \textit{stop words} mais comuns dos inglês e do 
português não foram indexadas. Além disso, somente palavras entre 3 e 30 caracteres
foram consideradas no índice. As listas invertidas de cada termo são comprimidas com
o método Elias-$ \delta $. Além das listas invertidas, o índice contém a norma dos vetores
dos documentos, a frequência do termo na coleção, o page rank de cada documentos 
e as listas invertidas dos \textit{anchor texts}. Sendo que, o texto dos \textit{anchor texts} 
aponta para o documento com a url do atributo "href".

O sistema de consultas implementado sempre recupera as listas invertidas completas dos termos
procurados e só considera no ranking a interseção das listas invertidas, ou seja, os documentos
que contém todos os termos procurados. Para palavras que aparecem em muitos documentos, esse
detalhe de implementação pode onerar um pouco a velocidade da máquina de busca. A menor lista 
invertida é recuperada primeiro e, para cada uma das listas invertidas posteriores, apenas os
documentos que já apareceram na primeira lista invertida são considerados. 

Antes de realizar as buscas, duas estruturas devem ser carregadas em memória principal: 
o dicionário e o mapa de documentos. O dicionário contém todos os termos, a posição das
suas respectivas listas invertidas no arquivo invertido, a posição das suas listas invertidas
relativas aos \textit{anchor texts} e a frequência do termo na coleção. O mapa de documentos guarda, 
para cada documento, a sua id, a sua url e o seu page rank. O carregamento destas estruturas 
demora cerca de 120 segundos no índice descrito na seção anterior. Uma vez carregado, as buscas
demoram apenas alguns milisegundos.

O sistema de consultas utiliza três modelos diferentes para realizar o ordenamento dos documentos
encontrados: o modelo vetorial, o modelo de anchor text e o page rank. Cada um destes modelos
possui um peso associado que multiplica o \textit{score} atribuído a cada documento. Estes pesos
servem para ajustar de forma linear a contribuição de cada modelo para o \textit{ranking} final.

\section{Modelo de espaço vetorial}

O modelo de espaço vetorial é o modelo base do nosso sistema de consultas. A medida que as
listas invertidas são analisadas, um acumulador é mantido para cada um dos documentos 
encontrados. Para cada termo, somamos a seguinte expressão ao acumulador:

\[
1 + log(f_{t,d}) \times log(\frac{N}{n_t})
\]

onde $ f_{t,d} $ é a frequência do termo no documento, que obtemos pela lista invertida, e
$ \frac{N}{n_t} $ é a frequência inversa do termo na coleção, que obtemos pelo dicionário. Ao
final da análise de todos os documentos, dividimos cada um dos acumuladores pela norma do
seu respectivo documento, que foi salva no índice durante a sua construção.

\section{Anchor text}

O modelo de \textit{anchor texts} funciona com base nos termos dos \textit{links}
encontrados na coleção. Cada termo encontrado nos \textit{anchor texts} dos \textit{links}
possui uma lista invertida como no caso do índice principal. No entanto, essa lista só contém 
a referência dos documentos apontados pelo \textit{link} onde aquele termo foi encontrado e um
contador para saber quantas vezes aquele termo apareceu apontando um dado documento. Quando um
\textit{link} apontava um documento fora da coleção, ele foi ignorado. O processo de construção
do índice de \textit{anchor texts} também serviu para construir o grafo do \textit{ Page Rank}, que
será descrito na próxima seção. O índice final de \textit{anchor ranks} ocupou cerca de 2GB.

O ranking por \textit{anchor texts} funciona de forma bastante simples. O \textit{score} atribuído a
cada documento é simplesmente o número de vezes que o termo da busca foi encontrado em \textit{anchor texts}
de \textit{links} que apontavam para o documento em questão. Para a maioria dos documentos, o \textit{score}
será zero. No entanto, em vários casos de busca, o \textit{anchor text} se mostrou uma ferramenta valiosa.

\section{Page Rank}

O \textit{Page Rank} é uma forma de atribuir relevância aos documentos independente dos termos buscados com base
no conceito de recomendação entre páginas da web. O \textit{Page Rank} foi calculado junto ao índice de \textit{anchor texts} 
em uma segunda passagem sobre a coleção após o índice principal já ter sido construído. Para cada \textit{link}
que apontava para um documento na coleção, foi salvo um ponteiro do documento de origem para o documento de destino. Inicialmente,
foi atribuído o valor 1 para cada documento da coleção e o algoritmo de \textit{Page Rank} foi executado por 300 iterações. A cada
iteração, o processo executado foi o seguinte: os documentos de origem distribuem seu \textit{score} para os documentos apontados 
pelos seus links. Documentos \textit{sink} (que não contém \textit{links} de saída) têm seu score acumulado e, ao final da iteração,
este score acumulado é distribuído entre todos os documentos da coleção. Por conta deste processo de distribuição, todos os documentos
da coleção acabam com um \textit{Page Rank} maior do que zero e, a soma dos \textit{Page Ranks} de todos os documentos é igual ao
número de documentos na coleção. Ao final do processo, o \textit{Page Rank} estima a relevância do documento na coleção, ou a 
probabilidade de um usuário estar navegando por ele caso nenhuma outra informação seja conhecida.

O \textit{Page Rank} se mostrou um sinal eficiente para melhorar os resultados das buscas, principalmente quando utilizado
em combinação com o índice de \textit{anchor texts}.

\section{Interface gráfica}

\section{Conclusão}

O objetivo deste trabalho era construir um índice invertido para a combinação das coleções obtidas no
Trabalho Prático 1 e esse índice foi construído com sucesso em um tempo total de 30 horas. O índice da coleção final
tem o tamanho de 16 GB (comprimido no formato tar.gz) e pode ser conferido em: \newline
visconde.latin.dcc.ufmg.br:/mnt/hd0/joao\_test/inverted\_index.data.tar.gz.

\end{document}
